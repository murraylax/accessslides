% This file is part of Access Slides. 
%
% Copyright (C) 2025 James M. Murray
%
% This program is free software: you can redistribute it and/or modify
% it under the terms of the GNU Affero General Public License as published
% by the Free Software Foundation, either version 3 of the License, or
% (at your option) any later version.
%
% This program is distributed in the hope that it will be useful,
% but WITHOUT ANY WARRANTY; without even the implied warranty of
% MERCHANTABILITY or FITNESS FOR A PARTICULAR PURPOSE.  See the
% GNU Affero General Public License for more details.
%
% You should have received a copy of the GNU Affero General Public License
% along with this program.  If not, see <https://www.gnu.org/licenses/>.

% --- PDF/UA metadata (new + fallback) ---
\ExplSyntaxOn
\keys_if_exist:nnTF { document / metadata } { tags }
  {
    \DocumentMetadata{
      pdfstandard = ua-1,
      pdfversion  = 1.7,
      tags,
      lang        = en-US,
      uncompress
    }
  }
  {
    \RequirePackage{pdfmanagement-testphase}
    \DocumentMetadata{
      testphase   = { tagging, uncompress },
      pdfstandard = ua-1,
      pdfversion  = 1.7,
      lang        = en-US
    }
  }
\ExplSyntaxOff
% In then next line, 'colortheme' is color theme in R,G,B (0-255) format
\documentclass[colortheme={8,92,150}, fontsize=18pt]{accessslides}

\title{Accessible Slides for LaTeX}
\author{James M. Murray, PhD}
\date{\today}

\begin{document}

\maketitle

\begin{slide}{Basic Slide}
  This is a basic slide. \\
  
  You can create slides using the \texttt{slide} environment: \\

  \texttt{\textbackslash begin\{slide\}\{<Slide Title>\}\\
  ... \\
  \textbackslash end\{slide\}} \\

  The content of each slide vertically-aligned to the top of the slide. Use \textbackslash\texttt{vspace*\{<length>\}} to adjust vertical position, if you wish. \\

  Must be compiled with \texttt{lualatex}.
\end{slide}

\begin{slide}{Basic Slide with Lists}
  \begin{enumerate}
    \item This is a numbered list item
    \item This is another numbered list item
      \begin{itemize}
        \item We can have bullet points too
        \item And more bullets
      \end{itemize}
  \end{enumerate}
\end{slide}

\subsection{Subsections Make Section-Separator Slides}

\section{Sections Mimic Regular Slides}

\begin{itemize}
  \item You can start a slide using \texttt{\textbackslash section\{\}} instead of \texttt{\textbackslash begin\{slide\}}
  \item This makes it behave well with Markdown
\end{itemize}

\begin{slide}{Columns}
  \begin{columns}
    \begin{column}{0.48\linewidth}
      You can make columns using the \texttt{columns} and \texttt{column} environments, much like Beamer.
    \end{column}
    \hfill
    \begin{column}{0.48\linewidth}
      \begin{itemize}
        \item Make sure to join your columns with \texttt{\textbackslash hfill} with no line spaces
      \end{itemize}
    \end{column}
  \end{columns}
\end{slide}

\begin{slide}{Columns with Blocks}
  \begin{columns}
    \begin{column}{0.48\linewidth}
      \begin{block}{Look at Me!}
        You can stand out using the \texttt{block} environment.
      \end{block}
    \end{column}
    \hfill
    \begin{column}{0.48\linewidth}
      \begin{block}{Another Block}
        You can have multiple blocks in a slide.
      \end{block}
    \end{column}
  \end{columns}

  \begin{block}{Yet Another Block}
    You can also have blocks that span the full slide width.
  \end{block}
\end{slide}

\begin{slide}{Include Graphics with \texttt{accgraphics}}
  \accgraphics[height=0.7\textheight]
  {Production possibilities frontier (PPF) illustrating that points inside the PPF are possible but inefficient, points on the frontier and possible and efficient, and points outside are not possible.}
  {./myimage.jpg}

  The \texttt{accgraphics} environment works like \texttt{includegraphics}, but requires alt text for accessibility.
\end{slide}

\begin{slide}{Decorative Graphics with \texttt{decographics}}

  \decographics[height=0.65\textheight]{./decorative.jpg}

  Use \texttt{decographics} instead of \texttt{accgraphics} for purely decorative images. They do not require alt text and are ignored by assistive technologies.
\end{slide}

\begin{slide}{Math}
  The \texttt{displaymath} environment is enhanced to optionally provide alt text for your math expressions. \\
  
  \begin{displaymath}[Sample standard deviation]
    s = \sqrt{\frac{\displaystyle\sum_{i=1}^{n} x_i^2 - \frac{1}{n} \left(\sum_{i=1}^{n} x_i\right)^2}{n - 1}}
  \end{displaymath}

  \ \\ \ \\
  The \texttt{equation} environment is similarly enhanced: 

  \begin{equation}[Euler's Identity]
    e^{i\pi} + 1 = 0 
  \end{equation}
\end{slide}

\begin{slide}{Simple With Minimal Features}
  \vspace*{-0.25in}
  \begin{itemize}
    \item This template is not nearly as full-featured as \href{https://latex-beamer.com/}{Beamer}.
    \item Many Beamer functions to sequentially reveal items are not accessible, and so are intentionally omitted.
    \item Other features like themes are omitted for simplicity and lack of time to work on it.
    \item Copyright 2025 James Murray, \href{https://www.gnu.org/licenses/gpl-3.0.en.html}{GNU General Public License v3.0}.
    \item This work is distributed in the hope that it will be useful, but WITHOUT ANY WARRANTY; without even the implied warranty of FITNESS FOR A PARTICULAR PURPOSE.
    \item I welcome bug reports, feedback, and suggestions, but I have very limited time to maintain this project. Do not expect responses, fixes, or new features in a timely manner.
  \end{itemize}
\end{slide}

\end{document}
